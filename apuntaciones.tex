\documentclass{article}
\usepackage[spanish]{babel}
\usepackage{times}
\usepackage[utf8]{inputenc}
%\usepackage[latin1]{inputenc}
\usepackage{amsmath}

%\newcommand{\em}[0]{\emph}

\begin{document}

PRÓLOGO

\small

\begin{flushright}

``Los españoles americanos, si dan todo el valor que dar 
se debe a la uniformidad de nuestro lenguaje en ambos 
hemisferios, han de hacer el sacrificio de atenerse, como 
a centro de uniformidad, al de Castilla, que le dio el ser 
y el nombre.''---Puighblanch

\end{flushright}

\normalsize

\paragraph{I.}

Es el bien hablar una de las más claras señales de la gente culta y bien nacida y condición 
indispensable de cuantos aspiren a utilizar en pro de sus semejantes, por medio de la palabra o 
de la escritura, los talentos con que la naturaleza los ha favorecido: de ahí el empeño con que 
se recomienda el estudio de la gramática. Pero siendo esta materia sobremanera abstrusa en la 
forma en que se explica en las obras relativas a ella y según se enseña en los colegios, tal 
que debe mirarse como ramo de alta filosofía, y siendo además esas obras insuficientes para lo 
que promete su definición por cuanto nada o casi nada nos dicen sobre la propiedad y pureza de 
las voces, acontece que los alumnos muy escaso provecho sacan de las aulas, y fuera de ellas 
pocos tienen el valor suficiente para consagrarse a aprenderla. Un libro, pues, escrito no en 
el estilo grave y estirado que demandan los trabajos didácticos, ni repleto de aquella balumba 
de reglas generalmente inútiles en la vida práctica por versar en su mayor parte sobre puntos 
en que nadie yerra; antes bien amenizado con todos los tonos y en el cual se contengan y 
señalen, digámoslo así, con el dedo las incorrecciones a que más frecuentemente nos deslizamos 
al hablar y al escribir, debe sin duda ser útil a los que no pueden vacar a estas 
especulaciones, de poca monta en apariencia, pero en realidad inaccesibles a la generalidad por 
la aplicación y muchos libros necesarios para ellas. Varias veces antes de ahora se ha 
acometido entre nosotros y con mayor o menor acierto llevádose a cabo esta empresa, y a 
satisfacer la misma necesidad nos hemos esforzado en estas Apuntaciones; sin la presunción de 
oscurecer a nuestros antecesores, reconocemos a cada cual su mérito, y confesamos serles 
deudores de observaciones que acaso se nos hubieran escapado.

Dichos sumariamente el motivo y objeto de esta obra, nos extenderemos algo más sobre su 
espíritu y el modo como hemos querido darle cima.

\paragraph{II.}

Nada, en nuestro sentir, simboliza tan cumplidamente a la Patria como la lengua: en ella se 
encarna cuanto hay de más dulce y caro para el individuo y la familia, desde la oración 
aprendida del labio materno y los cuentos referidos al amor de la lumbre, hasta la desolación 
que traen la muerte de los padres y el apagamiento del hogar; un cantarcillo popular evoca la 
imagen de alegres fiestas, y un himno guerrero, la de gloriosas victorias; en una tierra 
extraña aunque halláramos campos iguales a aquellos en que jugábamos de niños, y viéramos allí 
casas iguales a donde se columpió nuestra cuna, nos dice el corazón que, si no oyéramos los 
acentos de la lengua nativa, deshecha toda ilusión, siempre nos reputaríamos extranjeros y 
suspiraríamos por las auras de la Patria. De suerte que mirar por la lengua vale para nosotros 
tanto como cuidar los recuerdos de nuestros mayores, las tradiciones de nuestro pueblo y las 
glorias de nuestros héroes; y cuando varios pueblos gozan del beneficio de un idioma común, 
propender a su uniformidad es avigorar sus simpatías y relaciones, hacerlos uno solo. Por eso, 
después de quienes trabajan por conservar la unidad de creencias religiosas, nadie hace tanto 
por el hermanamiento de las naciones hispano-americanas, como los fomentadores de aquellos 
estudios que tienden a conservar la pureza de su idioma, destruyendo las barreras que las 
diferencias dialécticas oponen al comercio de las ideas. 

Pero ¿y cuál será la norma a que todos hayamos de sujetarnos? Ya que la razón no lo pidiera, la 
necesidad nos forzaría a tomar por dechado de nuestra lengua a la de Castilla, donde nació, y, 
llevando su nombre, creció y se ilustró con el cultivo de eminentísimos escritores, envidia de 
las naciones extrañas y encanto de todo el mundo; tipo único reconocido entre los pueblos 
civilizados, a que debe atenerse quien desee ser entendido y estimado entre ellos.

CAPÍTULO I.

ACENTUACIÓN.

GLOSARIO.

\small

\paragraph{} 1. \emph{Letra}: signo que representa un sonido, de ordinario elemental, de la voz humana; 
tambien se llama letra el sonido mismo.---Vocales son las letras que pueden pronunciarse por sí solas con 
claridad y distinción: a, e, i, o, u.---Consonantes son las que no pueden pronunciarse bien sin el 
auxilio de las vocales: como d, p, t.

\normalsize

\paragraph{} 5. Vamos a tratar en el presente capítulo de aquellas palabras en que arbitrariamente se ha 
cambiado el lugar del acento: descuellan entre estas muchas graves convertidas en esdrújulas a causa de 
la ignorancia de las lenguas sabias y de la pedantería de querer dar aire científico y campanudo a 
vocablos que en manera alguna han menester semejantes arreos. Apelando a la etimología y aduciendo 
ejemplos que patenticen la recta pronunciación, haremos comparecer los orondos esdrújulos en su ordinaria 
categoría de llanos, y las demás, en la forma que les corresponda; con lo cual quebraremos los ojos a 
quienes inconsultamente prohijan tales dislates.

\paragraph{} 6. En España principió esta invasión ridícula quizá antes que en nuestra patria; y si es 
cierto que los bogotanos pueden haber sacado algunos errores de esa fuente, deben también confesarse 
inventores de otros, y reconocer que en la Península han protestado los literatos contra semejante 
corruptela, cuándo con seriedad, cuándo donairosamente.Véanse algunas muestras de estas censuras.

``Hay también un \emph{neologismo fonético}, o de pronunciación, que desprecia los fundamentos de nuestra 
prosodia, y quebranta con todo el descaro de la insipiencia las leyes generales de acentuación 
castellana, casi siempre reflejo de la latina. Este neologismo prosódico es el que nos hace ya pronunciar 
\emph{análisis}\footnote{Bien sabemos que para probar la antigua pronunciación grave de este vocablo 
podría alegarse el soneto de Burguillos que comienza: ``Si cumplo con la lengua castellana''; pero aunque 
se pusiese de manifiesto que tal entonces era la práctica común y que posteriormente se introdujo la que 
hoy rige, en manera alguna abogaríamos por aquella, a causa de parecernos incorrecta: ora se consulten 
las reglas de la acentuación griega, ora las de la latina, tienen que ser esdrújulos \emph{análisis} y 
\emph{parálisis}: según aquella, porque la última sílaba es breve, según estotra, porque la penúltima lo 
es. Acaso tuvo presente el señor Monláu que la \emph{y} antes de \emph{s} es larga, pero hubo de olvidar 
que los verbales en \emph{sis} son precisamente excepción de la regla.---(Véase Anthon, \emph{A grammar 
of the Greek Language}, \emph{Prosody}, \emph{IX}, \emph{III}, \emph{7}.)

D. J. J. de Mora acentúa \emph{analísis}, \emph{paralísis} (\emph{Don Opas}, \emph{I}, \emph{LVII}); pero 
es sabido que este escritor aventura innovaciones prosódicas no siempre aceptables.}, \emph{fárrago}, 
\emph{médula}, \emph{parálisis}, etc; y si Dios y los eruditos no lo remedian acabará por hacernos decir 
\emph{cólega}, \emph{cónclave}, \emph{expédito}, \emph{intérvalo}, \emph{méndigo}, \emph{ópimo}, 
\emph{périto} y \emph{téstigo}.''---(D. Pedro Feipe Monláu, Del arcaísmo y el neologismo)

``Nunca he podido comprender, dice D. Eugenio de Ochoa, la general manía de convertir en esdrújulos 
vocablos que nunca lo han sido en castellano; y añadiré que esta manía, más que asombro, me causa 
envidia, pues se me figura, por ciertos indicios, que ha de ser, para el que está poseído de ella, 
ocasión de las más dulces sensaciones. Observo yo cierta fruición morosa en el retintín con que algunos 
pronuncian \emph{cólega}, en vez de \emph{colega}; \emph{intérvalo}, en vez de \emph{intervalo}. Hay 
quien parece que se va a a desmayar de gusto cuando dice que ha dado limosna a un \emph{méndigo}. Sobre 
este dislate, hoy tan común entre nosotros, solo me ocurre decir que le juzgo funesto, porque ataca de 
raíz el eufonismo de nuestra lengua, rompiendo la armoniosa proporción que debe existir entre las voces 
graves, agudas y esdrújulas de que se compone, y que constituye uno de sus más delicados primores.'' 
(\emph{París, Londres y Madrid}, \emph{pág. 559})

\small

\paragraph{} 7. La conservación del acento latino ha sido una de las reglas de formación de las lenguas 
romances: la sílaba acentuada constituye como el núcleo de la plabra: así de \emph{ministérium} han 
salido nuestro \emph{menestér}, el italiano \emph{mestiéro}, \emph{mestiére}, el provenzal 
\emph{menestiér}, \emph{mestiér}, el portugués \emph{mistér}, el francés \emph{metiér}; lo mismo, de
\emph{eleemósyna} se han formado el italiano \emph{limósina}, nuestro \emph{limósna}, antiguamente 
\emph{almósna} como en provenzal y reto-románico, el catalán y mallorquín \emph{almóyna}, el francés
\emph{aumône}, el portugués y gallego \emph{esmóla}.
\\

I

\normalsize

\paragraph{} 9. \emph{Académia\footnote{Para mayor claridad marcamos el acento en la sílaba a que llamamos 
la atención, aunque según las reglas ortográficas no deba marcarse}}: esta es la legítima pronunciación, no \emph{academía}. Ejemplos:

\small

\begin{verse}
Y si del ocio huyendo, por recreo \\
Busca la discreción de la \emph{académia}, \\
Que ser humilde tiene por trofeo, \\
Le sigue y le persigue la blasfemia, \\
Como si fuera público enemigo: \\
Tal es el precio conque el vulgo premia. \\
\end{verse}
\begin{flushright}
(Lupercio L. de Argensola, \emph{tercetos ``Obediente respondo'' etc.})
\end{flushright}

\begin{verse}
Mas ¿cómo tu \emph{académia} \\
No propone al divino Figueroa, \\
Si con verde Laurel sus hijos premia?
\end{verse}
\begin{flushright}
(Lope, \emph{Laurel de Apolo, silva IV.})
\end{flushright}

\begin{verse}
A las conversaciones y \emph{académias} \\
Donde los ambiciosos, \\
De opinión y títulos famosos, \\
Con aplauso comprado \\
Leen el libro o poema meditado, \\
No vayas imprudente, \\
Ni llamado te llegues fácilmente.
\end{verse}
\begin{flushright}
(Quevedo, \emph{Doctr. de Epict., cap. XXXVI.})
\end{flushright}

\begin{verse}
Escuela de las traiciones \\
Y \emph{académia} de los vicios.
\end{verse}
\begin{flushright}
(Calderón, \emph{La vida es sueño, jorn. I.})
\end{flushright}

Se ha dudado si en griego la penúltima sílaba es \emph{i} o \emph{ei}, 
pero los lugares poéticos donde ocurre el vocablo han resuelto el punto 
en favor del diptongo. Sin necesidad de esto se habría llegado a la misma
conclusión con ver lo que pasa en latín: Cicerón (\emph{Divin.} 1, 13, 22)
alarga la \emph{i}, en tanto que Claudiano (\emph{Cons. Mall. Theod.} 94)
y Sidonio Apolinar (\emph{Epithal. Polem.} 153) la abrevian. Por aquí se ve
que dicha \emph{i} representa el diptongo \emph{ei} que hace retroceder
el acento. En tiempo de Cicerón, cuando la cantidad predominaba sobre el
acento, se podía pronunciar una penúltima larga sin ser acentuada; en la 
decadencia, cuando la cantidad cedía al acento, el de la sílaba \emph{de}
hizo que se abreviase la \emph{i}. Si originariamente se hubiera hallado
en griego el acento en la \emph{i}, hubiera sucedido lo contrario, según se 
observó en el \S{} 7 con respecto a \emph{sophia}. En castellano, pues, se ha 
conservado la legítima acentuación. Este es uno de aquellos casos en que las
lenguas romances dan luz para resolver puntos oscuros de las antiguas.
El acentuar la \emph{i} no es, sin embargo, práctica reciente: entre otros
lo hizo Solís en la Silva que empieza: \emph{¿Campana, y a estas horas ...?}
Véase Cadalso, \emph{Cartas marruecas, LXVII}.

\normalsize

\paragraph{} 10. El sufijo latino \emph{monia}, \emph{monio} lleva constantemente en 
castellano el acento sobre la \emph{o}: \emph{parsimónia}, \emph{santimónia}, 
\emph{ceremónia}; \emph{matrimónio}, \emph{património}, \emph{testimonio}; la misma acentuación corresponde, pues, a \emph{acrimónia}, formado de \emph{acre}.

\small

\begin{verse}
Dormiré bien y criaré buen quilo, \\
Templaré la \emph{acrimónia} de la bilis.
\end{verse}
\begin{flushright}
(D. Tomás de Iriarte, \emph{Epíst, III.})
\end{flushright}

\begin{verse}
Y aun con mayor \emph{acrimónia}, \\
Probó el poeta Menandro \\
Que, aunque nació en Macedonia \\
El magnánimo Alejandro \\
Fue colegial de Bolonia
\end{verse}
\begin{flushright}
(Id., \emph{Quintillas disparatadas.})
\end{flushright}

En nombres como \emph{acedía}, \emph{bizarría}, el sufijo es \emph{ía}, y por consiguiente no pueden tomarse
como norma para \emph{acrimonia}.

\paragraph{} 11. \emph{Cólega} debe pronunciarse \emph{colega}, y \emph{concólega}, \emph{concolega}.
Ejemplo:

\begin{verse}
Tribuno Cota, viendo los alientos \\
Y errores del \emph{coléga} licencioso, \\
Mal conducido a términos sangrientos, \\
Le aconseja sagaz, no temeroso.
\end{verse}
\begin{flushright}
(Jáuregui, \emph{Farsalia, lib. V.})
\end{flushright}

Trae su origen esta voz del latín \emph{collega}, compuesto de la preposición \emph{cum} y de \emph{legare},
diputar: este, como inmediatamente conexo con \emph{lex} 
\footnote{Véase Pott, \emph{Wurzel-Wörterbuch der Indo-Germanischen Sprachen, tomo III, pág. 609}; Vanicek,
\emph{Griechisch-Lateinisches etymologisches Wörterbuch, pág. 833.}}, 
tiene la primera sílaba larga, de donde \emph{collega} tiene igual cantidad en la penúltima, y por lo tanto 
viene a ser grave.

\normalsize

\paragraph{} 12. Dícese \emph{dominíco} por \emph{dominicano}, a diferencia de \emph{domínico}, adjetivo que 
significaba lo propio del Señor. En todas las ediciones del Dicionario de la Academia hasta la 10a. inclusive
se lee \emph{dominica} (domingo) sin acento; en las posteriores está como esdrújulo.

\small

\begin{verse}
Su padre, como era rico, \\
Le crió en ostentación, \\
Mas el mozo desde chico, \\
Tuvo siempre inclinación \\
A ser fraile \emph{dominíco}.
\end{verse}
\begin{flushright}
(Cáncer y Velasco, \emph{Obras varias,} fol. 35 vo: Madrid, 1651.)
\end{flushright}

El mismo Cáncer acentúa \emph{dominíca}, fol. 27; también Torres Naharro, \emph{Propaladia, tomo II, página 264}:
Madrid, 1900.

\normalsize

\paragraph{} 13. \emph{Elefancía} se lee en la Silva de consonantes de Rengifo, y así acentúa la Academia en todas
las ediciones de su Diccionario; otros como Gracia (Aicart) y Peñalver en los que escribieron de la rima, 
pronuncian lo mismo que todos nuestros coterráneos, acomodando el vocablo a la acentuación de las numerosas
voces latinas en \emph{-ancia}. Dicho se está que debemos arrimarnos a la primera autoridad.

\small

Por no hallarse esta voz en los diccionarios griegos ni en verso latino alguno, no se puede fijar la acentuación
originaria; es posible que haya seguido la analogía delos acabados en \emph{mancía} (adivinación), como 
\emph{nigromancía}, \emph{quiromancía}, etc. He aquí algunos ejemplos que comprueban la acentuación de estos
últimos vocablos:

\begin{verse}
Estudié \emph{nigromancía}, \\
Como te he dicho, en Granada.
\end{verse}
\begin{flushright}
(Lope, \emph{El servir con mala estrella, acto II,esc. XII.})
\end{flushright}

\begin{verse}
Lo que es\emph{lecanomancía}, \\
Que se hace en agua, y adonde \\
El espíritu responde, \\
Topéla en el Plinio un día.
\end{verse}
\begin{flushright}
(Id., \emph{Servir a señor discreto, acto II,esc. IX.})
\end{flushright}

\normalsize

\paragraph{} 309. Hablando de una señora decia un periódico que “se le veia siempre en los hospitales.” Este es 
yerro que debe cuidadosamente evitarse, pues en estas construcciones de sentido impersonal se dice siempre en el 
femenino la, las, segun que lo atestiguan los ejemplos siguientes:
eso
``Esta unidad es tan esencial en esta clase de composiciones como en todas las obras de bellas artes; el drama 
mas nutrido de sucesos la consiente, ó, por mejor decir, la exige, así como se la admira en los inmensos cuadros 
de Julio Romano.'' (Martínez de la Rosa, Apuntes sobre el drama histórico.)--- ``El  ama bonita suele gastar 
buen genio, pues como se la mima y regala, no hay motivo para que se le exalte la bílis.'' (Hartzenbusch, El ama 
de llaves.)

\paragraph{} 310. Acabamos de ver que en las construcciones de sentido impersonal, se dice se la, se las, cuando 
se habla de mujeres: hablándose de hombres, se dice se le, se les; ejemplos:

“Se convierte á Alejandro (en el poema de su nombre) en un paladin de la edad média, y se le arma caballero con 
todas las formalidades que entónces se usaban.” (Gil y Zárate, Manual de literatura, pte. II, seccion I, cap. 
II.) “Los prosistas quedan por lo regular confinados en las bibliotecas, de donde no se les saca sino de cuando 
en cuando para consultarlos.” (Id, ib, seccion II, cap. ) “Faltos los más de la competente instrucción, se les 
ve incurrir á veces en errores manifiestos, como los que notó el sensato Luzan áun en los autores de más fama.” 
(Martínez de la Rosa, Apuntes sobre el drama histórico.): 

\paragraph{} 331. Con poco acuerdo se confunden generalmente las expresiones el mismo, uno mismo: la primera 
presupone siempre un término de comparación, o en lo que precede, o en lo que sigue, cosa que no sucede con la 
segunda. Con ejemplos se esclarecerá esta diferencia, que a algunos pudiera ser sutil y caprichosa, pero 
sustentada por la práctica de los escritores más correctos y castizos y por el valor de los elementos que 
constituyen ese modo de hablar.
“Mientras que en la corte se hacían estas tentativas tan vanas como viles para destruir al maestre, los grandes 
por su parte, aunque desparramados y dispersos se entendían y confederaban en la misma intención” (Quintana, 
Vida de don Álvaro de Luna); esto es, “en la intención de destruir al maestre”, de la cual se habló primero.
“Con la misma lengua y las mismas palabras que usa el palurdo, hablan el sabio y el orador” (Capmany, Filosofía 
de la elocuencia, prólogo): aquí la palabra que demuestra que se trata de una comparación; si se suprimiese y en 
lugar de dos verbos se pusiese solo uno, era menester decir uno mismo: “El palurdo, el sabio y el orador hablan 
con una misma lengua y unas mismas palabras”.
Como en vez de el mismo nadie emplea uno mismo, sino que, al contrario, se ignora el uso de este, allegaremos 
unos tantos ejemplos que muestren los casos en que es forzoso su uso:

“El hombre nacido de mujer vive poco tiempo, está lleno de muchas miserias; sale como una flor, y luego se 
marchita, y huye como sombra, y nunca permanece en un mismo estado.” (Fray Luis de Granada, Guía de pecadores, 
lib. I, cap. VIII.) – Nombró cardenales en un mismo día dos sobrinos suyos.” (Mariana , Historia de España, lib. 
XXII, cap. XVII.) – No todas las cosas suceden de un mismo modo (Cervantes, Quij., pte. I, cap. XIX.)

Por aquí se ve lo que arriba indicamos, que uno mismo no presupone un término de comparación ni en lo que 
precede ni en lo que sigue, pues las frases en que entra ofrecen un sentido completo y cabal.
En lugar de uno mismo se dice muy elegantemente uno; v. gr.

\paragraph{} 440. Vamos a tratar del grande escollo no solo de los bogotanos sino de la mayor 
parte de los americanos, del \emph{que} galicado por excelencia, del \emph{que} contrapuesto mediante el 
verbo \emph{ser} a adverbios y complementos. No contento con bizarrear en los escritos de los periodistas, 
poetastros, filosofastros y la innúmera caterva de los demás corruptores de la lengua 
castellana, y aun en los de autores por otra parte estimables, va cundiendo anchamente en el 
lenguaje familiar y aun en el vulgar. Varias veces se ha dado la voz de alarma, pero, según 
parece, muy pocos entre el común de los lectores han caído en el chiste, y no conociendo el famoso 
\emph{que}, consiguientemente no saben evitarlo\footnote{En prueba de esto léase el siguiente pasaje de
una obra de crítica gramatical que nos daría pena citar con sus pelos y señales: ``Cuando en cierto
escrito leímos: `\emph{Es por eso} que no me complace la lectura de los clásicos', dijimos: Verdad; 
porque, a gustarle, habría dicho: \emph{Por eso es} que no me gusta la lectura de los clásicos.'' El 
crítico achaca aquí el error a la transposición.---Para descargo de nuestra conciencia y para curarnos en 
salud, advertimos que la aplicación burlesca del adjetivo \emph{galicado} $=$ \emph{galicoso} que hacemos 
aquí y acaso en otras ocasiones, es ocurrencia de Moratín. El vocablo se halla en el diccionario de 
Salvá, y ha sido aceptado recientemente por la Academia.}; por este motivo vamos a presentar muestras de 
él 
con los varios giros que pueden usarse en su reemplazo, valiéndonos para ello de algunas frases 
francesas con su versión.
\\

1. \emph{Ce fut dans le XV siècle} \textsc{que} \emph{l'Amérique fut découverte.}

Traducci\'on bárbara:

\emph{Fue en el siglo XV} \textsc{que} \emph{se descubrió la América}.

Como se ve, se ha dejado el \emph{que} del francés contrapuesto al complemento \emph{en el siglo XV}; para que 
eso sea castellano es menester poner en lugar del \emph{que} solo, un complemento análogo al 
anterior:

\emph{Fue en el siglo XV} \textsc{en el que} \emph{se descubrió América};

o poner el adverbio correspondiente, que, hablándose de tiempo, será \emph{cuando}:

\emph{Fue en el siglo XV} \textsc{cuando} \emph{se descubrió América}.

Todavía tienen cabida otros modos: v. gr.

\emph{El siglo XV  fue} \textsc{el en que} \emph{se descubrió América};

\emph{El siglo XV  fue} \textsc{en el que} \emph{se descubrió América};

\emph{El siglo XV  fue} \textsc{cuando} \emph{se descubrió América}.

Puede simplificarse quitando el verbo \emph{ser} y el relativo y formando de dos frases una:

\emph{En el siglo XV se descubrió América};
\\
pero como con esta simplificación se pierde en ocasiones lo enfático de los anteriores giros, puede compensarse, 
cuando fuere necesario, con la adicion de otra palabra, como \emph{precisamente}, 
\emph{cabalmente}:

\emph{Precisamente en el siglo XV se descubrió América}.

\end{document}