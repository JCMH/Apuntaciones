\documentclass{book}

\begin{document}

PROLOGO

“Los españoles americanos, si dan todo el valor que dar 
se debe á la uniformidad de nuestro lenguaje en ambos 
hemisferios, han de hacer el sacrificio de atenerse, como 
á centro de uniformidad, al de Castilla, que le dió el sér 
y el nombre.”---Puighblanch

I.

Es el bien hablar una de las más claras señales de la gente culta y bien nacida y condicion indispensable de cuantos aspiren á utilizar en pro de sus semejantes, por medio de la palabra ó de la escritura, los talentos con que la naturaleza los ha favorecido: de ahí el empeño con que se recomienda el estudio de la gramática. Pero siendo esta materia sobremanera abstrusa en la forma en que se explica en las obras relativas á ella y según se enseña en los colegios, tal que debe mirarse como ramo de alta filosofía, y siendo ademas esas obras insuficientes para lo que promete su definicion por cuanto nada ó casi nada nos dicen sobre la propiedad y pureza de las voces, acontece que los alumnos muy escaso provecho sacan de las aulas, y fuera de ellas pocos tienen el valor suficiente para consagrarse á aprenderla. Un libro, pues, escrito no en el estilo grave y estirado que demandan los trabajos didácticos, ni repleto de aquella balumba de reglas generalmente inútiles en la vida práctica por versar en su mayor parte sobre puntos en que nadie yerra; ántes bien amenizado con todos los tonos y en el cual se contengan y señalen, digámoslo así, con el dedo las incorrecciones á que más frecuentemente nos deslizamos al hablar y al escribir, debe sin duda ser útil á los que no pueden vacar á estas especulaciones, de poca monta en apariencia, pero en realidad inaccesibles á la generalidad por la aplicacion y muchos libros necesarios para ellas. Varias veces ántes de ahora se ha acometido entre nosotros y con mayor ó menor acierto llevádose a cabo esta empresa, y á satisfacer la misma necesidad nos hemos esforzado en estas Apuntaciones; sin la presuncion de oscurecer á nuestros antecesores, reconocemos á cada cual su mérito, y confesamos serles deudores de observaciones que acaso se nos hubieran escapado.

Dichos sumariamente el motivo y objeto de esta obra, nos extenderemos algo más sobre su espíritu y el modo como hemos querido darle cima.

II.

Nada, en nuestro sentir, simboliza tan cumplidamente á la Patria como la lengua: en ella se encarna cuanto hay de más dulce y caro para el individuo y la familia, desde la oracion aprendida del labio materno y los cuentos referidos al amor de la lumbre, hasta la desolacion que traen la muerte de los padres y el apagamiento del hogar; un cantarcillo popular evoca la imágen de alegres fiestas, y un himno guerrero, la de gloriosas victorias; en una tierra extraña aunque halláramos campos iguales á aquellos en que jugábamos de niños, y viéramos allí casas iguales á donde se columpió nuestra cuna, nos dice el corazon que, si no oyéramos los acentos de la lengua nativa, deshecha toda ilusion, siempre nos reputaríamos extranjeros y suspiraríamos por las auras de la Patria. De suerte que mirar por la lengua vale para nosotros tanto como cuidar los recuerdos de nuestros mayores, las tradiciones de nuestro pueblo y las glorias de nuestros héroes; y cuando varios pueblos gozan del beneficio de un idioma comun, propender á su uniformidad es avigorar sus simpatías y relaciones, hacerlos uno solo. Por eso, despues de quienes trabajan por conservar la unidad de creencias religiosas, nadie hace tanto por el hermanamiento de las naciones hispano-americanas, como los fomentadores de aquellos estudios que tienden á conservar la pureza de su idioma, destruyendo las barreras que las diferencias dialécticas oponen al comercio de las ideas. 

Pero ¿y cuál será la norma á que todos hayamos de sujetarnos? Ya que la razon no lo pidiera, la necesidad nos forzaria á tomar por dechado de nuestra lengua á la de Castilla, donde nació, y, llevando su nombre, creció y se ilustró con el cultivo de eminentísimos escritores, envidia de las naciones extrañas y encanto de todo el mundo; tipo único reconocido entre los pueblos civilizados, á que debe atenerse quien desee ser entendido y estimado entre ellos.






































CAPITULO I.

ACENTUACION.

GLOSARIO.

1. Letra: signo que representa un sonido, de ordinario elemental, de la voz humana; tambien se llama letra el sonido mismo.---Vocales son las letras que pueden pronunciarse por sí solas con claridad y distincion: a, e, i, o, u.---Consonantes son las que no pueden pronunciarse bien sin el auxilio de las vocales: como d, p, t.

5. Vamos á tratar en el presente capítulo de aquellas palabras en que arbitrariamente se ha cambiado el lugar del acento: descuellan entre éstas muchas graves convertidas en esdrújulas á causa de la ignorancia de las lenguas sábias y de la pedantería de querer dar aire científico y campanudo á vocablos que en manera alguna han menester semejantes arreos. Apelando á la etimología y aduciendo ejemplos que patenticen la recta pronunciacion, haremos comparecer los orondos esdrújulos en su ordinaria categoría de llanos, y las demas, en la forma que les corresponda; con lo cual quebraremos los ojos á quienes inconsultamente prohijan tales dislates.



309. Hablando de una señora decia un periódico que “se le veia siempre en los hospitales.” Este es yerro que debe cuidadosamente evitarse, pues en estas construcciones de sentido impersonal se dice siempre en el femenino la, las, segun que lo atestiguan los ejemplos siguientes:

“Esta unidad es tan esencial en esta clase de composiciones como en todas las obras de bellas artes; el drama mas nutrido de sucesos la consiente, ó, por mejor decir, la exige, así como se la admira en los inmensos cuadros de Julio Romano.” (Martínez de la Rosa, Apuntes sobre el drama histórico.)--- “El  ama bonita suele gastar buen genio, pues como se la mima y regala, no hay motivo para que se le exalte la bílis.” (Hartzenbusch, El ama de llaves.)

310. Acabamos de ver que en las construcciones de sentido impersonal, se dice se la, se las, cuando se habla de mujeres: hablándose de hombres, se dice se le, se les; ejemplos:

“Se convierte á Alejandro (en el poema de su nombre) en un paladin de la edad média, y se le arma caballero con todas las formalidades que entónces se usaban.” (Gil y Zárate, Manual de literatura, pte. II, seccion I, cap. II.) “Los prosistas quedan por lo regular confinados en las bibliotecas, de donde no se les saca sino de cuando en cuando para consultarlos.” (Id, ib, seccion II, cap. ) “Faltos los más de la competente instrucción, se les ve incurrir á veces en errores manifiestos, como los que notó el sensato Luzan áun en los autores de más fama.” (Martínez de la Rosa, Apuntes sobre el drama histórico.): 

331. Con poco acuerdo se confunden generalmente las expresiones el mismo, uno mismo: la primera presupone siempre un término de comparación, o en lo que precede, o en lo que sigue, cosa que no sucede con la segunda. Con ejemplos se esclarecerá esta diferencia, que a algunos pudiera ser sutil y caprichosa, pero sustentada por la práctica de los escritores más correctos y castizos y por el valor de los elementos que constituyen ese modo de hablar.
“Mientras que en la corte se hacían estas tentativas tan vanas como viles para destruir al maestre, los grandes por su parte, aunque desparramados y dispersos se entendían y confederaban en la misma intención” (Quintana, Vida de don Álvaro de Luna); esto es, “en la intención de destruir al maestre”, de la cual se habló primero.
“Con la misma lengua y las mismas palabras que usa el palurdo, hablan el sabio y el orador” (Capmany, Filosofía de la elocuencia, prólogo): aquí la palabra que demuestra que se trata de una comparación; si se suprimiese y en lugar de dos verbos se pusiese solo uno, era menester decir uno mismo: “El palurdo, el sabio y el orador hablan con una misma lengua y unas mismas palabras”.
Como en vez de el mismo nadie emplea uno mismo, sino que, al contrario, se ignora el uso de este, allegaremos unos tantos ejemplos que muestren los casos en que es forzoso su uso:

“El hombre nacido de mujer vive poco tiempo, está lleno de muchas miserias; sale como una flor, y luego se marchita, y huye como sombra, y nunca permanece en un mismo estado.” (Fray Luis de Granada, Guía de pecadores, lib. I, cap. VIII.) – Nombró cardenales en un mismo día dos sobrinos suyos.” (Mariana , Historia de España, lib. XXII, cap. XVII.) – No todas las cosas suceden de un mismo modo (Cervantes, Quij., pte. I, cap. XIX.)

Por aquí se ve lo que arriba indicamos, que uno mismo no presupone un término de comparación ni en lo que precede ni en lo que sigue, pues las frases en que entra ofrecen un sentido completo y cabal.
En lugar de uno mismo se dice muy elegantemente uno; v. gr.

413. Vamos a tratar del grande escollo no sólo de los bogotanos sino de la mayor parte de los americanos, del que galicado por excelencia, del que contrapuesto mediante el verbo ser á adverbios y complementos: no contento con bizarrear en escritos de los periodistas, poetastros, filosofastros y la innúmera caterva de los demas corruptores de la lengua castellana, y áun en los de autores por otra parte estimables, va cundiendo anchamente en el lenguaje familiar y áun en el vulgar. Varias veces se ha dado la voz de alarma, pero, segun parece, muy pocos entre el comun de los lectores han caido en el chiste, y no conocen el famoso que, y consiguientemente no saben evitarlo;1 por este motivo vamos á presentar muestras de él con los varios giros que pueden usarse en su reemplazo, para lo cual pondremos algunas frases francesas con su version.

1. Ce fut dans le XV siècle que l'Amérique fut découverte.

Traduccion bárbara:

Fué en el siglo XV que se descubrió la América.

Como se ve, se ha dejado el que del frances contrapuesto al complemento en el siglo XV; para que eso sea castellano es menester poner en lugar del que solo, un complemento análogo al anterior:

Fué en el siglo XV en el que se descubrió América;

ó poner el adverbio correspondiente, que, hablándose de tiempo, será cuando:

Fué en el siglo XV cuando se descubrió América.

Todavía tienen cabida otros modos: v. gr.

El siglo XV  fué  el en que se descubrió América;
El siglo XV  fué  en el que se descubrió América;
El siglo XV  fué  cuando se descubrió América.

Puede simplificarse quitando el verbo ser y el relativo y formando de dos frases una:

En el siglo XV se descubrió América;

pero como con esta simplificacion se pierde en ocasiones lo enfático, puede compensarse, cuando fuere necesario, con la adicion de otra palabra como precisamente, cabalmente:

Precisamente en el siglo XV se descubrió América.

\end{document}